\section{Summary}
In this project, I have successfully investigated the applicability of generalised planning in complex path finding games, as well as created a tool in which planning instances can be created and customised. In this, I explored a variety of games: Maze and Snake. When creating these, I realised that they were far too complex for the generalised planner to find solutions, and hence performed a problem reduction. This led to a significant improvement in the synthesis of generalised plans. In addition to the creation of problems, I also evaluated the effect of larger environments, where the agent must travel further to reach a goal. In this, I discovered that the generalised planner did not perform as well as the classical planner for very large environments. In a third and final experiment, I also explored the effect of increasing the length of the algorithm created in a generalised plan, to determine whether this would generate better classical plans. This did not meet expectations, as there was no correlation between the length of the program and cost of the classical plans. To end this summary, I also gained a wide variety of knowledge on knowledge-based agents as well as gaining a deeper insight into Automated Planning as a field.

\newpage
\section{Achievements}
The initial aims and objective have been met as well as exceeded with the introduction of problem reductions. The below table lists the objectives and how they were met in this project.

\begin{table}[ht]
\centering
\begin{tabular}{|p{0.473\linewidth}|p{0.473\linewidth}|}
\hline
Objective & How it was achieved \\\hline
Explore state-of-the-art planners used in Automated Planning in recent years & I have researched into the history of planners and knowledge-based agents, and utilised two planners that are currently well known and appreciated for being the best in the field. Furthermore, I have gained more knowledge on first-order-logic which has its importance outside of AI. \\\hline
Model path finding problems (Maze and Snake) as planning problems & I was able to model both the Maze and Snake problems, as well as providing a problem reduction on the Maze problems that yield more optimal results.
\\\hline
Create easy-to-understand visuals that displays generated path finding problems & I created an interactive display for the Maze generator, where the user can create custom Maze problems. However I was not able to do this for the Snake problems.\\\hline
Perform analysis on the performance of generalised planning on the created path finding problem instances & As described in the Experiments section, I was able to evaluate a variety of cases in which the generalised planner performed well and where it has its limitations.
\\\hline
Compare the plans generated by classical plans and generalised plans & Finally, I was able to measure the generalised plan's performance against a set of classical plans, shown in the Experiments section.
\\\hline
\end{tabular}
\caption{Goals Achieved}
\end{table}
\noindent Overall, I am proud of this project, as it was able to meet all the criteria I set. Furthermore, the knowledge gained from this is invaluable as the field of AI is ever growing.

\section{Critical Reflections}
Although all the criteria in this project's objectives was met, there are several areas for improvement. Currently the generation of Mazes still have some bugs in them, although usable, sometimes when setting the start and goal nodes, if the user wants to change the start node, sometimes it causes the program to crash. Furthermore, I would've liked to implement a manual generation option for the Snake games, had more time been available.  A final point to reflect on is that, if I were to re-do this project, I would plan the structure a lot earlier before developing. By following a strict iterative approach after having programmed for a couple months set me moderately far behind, potentially reducing the scale of this project. A better plan would have involved thinking about the core of the program first, which was the problem generator. Despite this, I am satisfied that the generalised planner is able to solve the problems I have created, as well as the optimisations made within the Maze domain, as this is the main focus of the investigation of this project.

\section{Future Work}
Following the research done by this project, in future works, I would like to extend the problem reduction to a variety of different domain models, potentially devising a solution that can be applied to any domain model. In terms of the visual aspects of this project, I would like to develop a more sophisticated user interface that allows the user to create new problems as well as being able to view the plans in real time. Furthermore, I would like to implement other games in the Blockly series such as the Bird game that involves the agent moving a fixed distance over a specific orientation. Overall, the most of the proposed future developments would involve making the project look better, which is a testament to the goals achieved in the development of this project.