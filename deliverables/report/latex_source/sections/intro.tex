\section{Motivation and Context}
Path finding in games is a topic that has been researched by many researchers in recent history. It is a complex problem that involves the use of a variety of algorithms. Techniques used in path finding are not limited to games, and can be used in a variety of fields within Artificial Intelligence (AI), such as robotics and logistics. Generalised planning is a relatively new field in AI, traditionally focusing on solvable instances in the synthesis of programs (2003)\cite{jimenez_review_2019} to uses in robotics seen since the previous decade. It involves complex theory to solve a collection of planning instances in a single algorithm-like plan. These problems share the same observations and actions from some real world \textit{environment}. Computing this algorithm gives the individual solutions to the given planning instances. Due to its novelty, this project aims to expand the horizons by utilising these planners in gaming notably simple path finding games, evaluating its performance and its limitations. \\\\
Another motivation behind this project is to develop a series of \textit{complex path finding problems} that test the limits of generalised planners developed in the recent years. This stems from the lack of existing tools that utilise generalised planning, as the focus is usually on \textit{classical planning}. These problems must developed in a way such that existing planners can solve these problems in the most efficient way possible. This project is aimed at students and anyone who are looking to further their knowledge in the field of Knowledge-Based AI. Although the primary focus of this project is generalised planning, as classical planning is relevant to the field, it is also covered widely within this report. 


\newpage
\section{Aims and Objectives}
Due to the complexities involved in generalised planning, the goal of this project is to provide a easy-to-use tool with optimised planning examples, that students can use, learn and also develop from. Initially, this project focused only on Maze games, however this was extended to evaluate a variety of path finding games. Below are the objectives that have been outlined before the development of this project:

\begin{table}[ht]
\centering
\begin{tabular}{|p{0.473\linewidth}|p{0.473\linewidth}|}
\hline
Objective & What it involves \\\hline
Explore state-of-the-art planners used in Automated Planning in recent years & This involves researching both classical planners and generalised planners. \\\hline
Model path finding problems (Maze and Snake) as planning problems & This involves first creating the path finding problem in some format, then converting it into some instance that can be solved using a planner.
\\\hline
Create easy-to-understand visuals that displays generated path finding problems & This also involves creating interactive displays in which the user can create solvable path finding problem instances. \\\hline
Perform analysis on the performance of generalised planning on the created path finding problem instances & This involves measuring the impact of the size of paths created, or other various variables induced on the creation of plans from path finding problem instances.
\\\hline
Compare the plans generated by classical plans and generalised plans & This involves an investigation that finds answers the question: \textit{in which situations does one planner perform better than the other?}
\\\hline
\end{tabular}
\caption{Aims and Objectives}
\end{table}

\section{Report Structure}
This report starts with a brief introduction involving the motivation behind this project as well as the objectives that must be met in the finished product. The \textbf{2. Background} section explores the technical knowledge required to understand the report fully. This will include the history of planning starting from knowledge-based agents, to modern generalised planning. Then in the \textbf{3. Design and Implementation} section, a description of the design used in the creation this project will be given, including the architecture and \textit{problem generators}. The \textbf{4. Experiments and Evaluation} section will cover a variety of experiments that test the applicability of planners on the create path finding instances. Finally the report will conclude with the \textbf{5. Summary and Conclusions} section that will critically analyse the project as a whole, as well as listing the achievements met by this project and any future work that could improve the final design.